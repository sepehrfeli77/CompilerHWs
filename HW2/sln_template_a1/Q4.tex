%نام و نام خانوادگی:
%شماره دانشجویی: 
\مسئله{پرانتزگذاری معتبر}

\پاسخ{
بله 
\lr{LL(1)}
 است.
\newline
\lr{
		\begin{table}[H]
			\begin{tabular}{c|c|c}
				 & First   & Follow \\
				\hline
	S & ( $\epsilon$ & \$ )\\
	P & ( & \$ ( \\
			\end{tabular}
		\end{table}
	}
	
\lr{
		\begin{table}[H]
			\begin{tabular}{c|c|c|c}
				 & (  & ) & \$ \\
				\hline
	S & S $\rightarrow$ PS & S $\rightarrow$ $\epsilon$ & S $\rightarrow$ $\epsilon$ \\
	\hline
	P & P $\rightarrow$ (S) & &  \\
			\end{tabular}
		\end{table}
	}
\newline
دو نوع خطا ممکن است در گرامر‌های
\lr{LL(1)}
 رخ دهد.
\begin{itemize}
\item
  اولی اینکه هنگام پارس به در قسمت پیش‌بینی(سمت چپ) به 
 \$
 برسیم ولی هنوز کل رشته‌را پارس نکرده باشیم.(در سمت راست هنوز رشته برای پارس وجود دارد اما سمت چپ خالی است.) 
\newline
برای گرامر این سوال می‌توان رشته
\lr{((}
 را مثال زد.

\item
نوع دیگر خطا حالتی است که در یک محصول برای ترمینال بعدی که می‌بینیم در جدول خالی باشد. در این حالت پارس متوقف می‌شود.
\newline
مثال: 
این حالت برای این سوال اتفاق نمی‌افتد.
\end{itemize}
}