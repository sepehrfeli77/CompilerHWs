%نام و نام خانوادگی:
%شماره دانشجویی: 
\مسئله{ }

\پاسخ{
\begin{enumerate}
	\item
		\begin{latin}
			\begin{table}[H]
				\begin{tabular}{c|c}
					Instruction & Live Variables\\
					\hline
					a = 1 + 2; & {b, e, f} \\
					b = a + b; & {a, b, e, f}\\
					z = a * 2; & {a, b, e, f}\\
					c = b + e; & {a, b, e, f}\\
					d = c + b; & {a, b, c, f}\\
					x = b + 3; & {a, b, c, d, f}\\
					z = a * 8; & {a, c, d, f, x}\\
					t = c - 2;  & {c, d, f, x, z}\\
					f = x + f;  & {d, f, x, z}\\
					y = x - 2;  & {d, x, z}\\
					d = d - y;  & {d, x, y, z}\\
							& {d, x, z}\\
				\end{tabular} 
			\end{table} 
		\end{latin}
	\item
در دور اول، دستورات زیر به ترتیب حذف میگردند. (با توجه به آنالیز متغیرهای زنده بخش قبلی)
		\begin{enumerate}
			\item
ابتدا دستور \lr{f = x + f;} حذف میگردد. زیرا بعد از آن، متغیر f زنده نخواهدبود.
			\item
سپس دستور \lr{t = c - 2;} حذف میگردد. زیرا بعد از آن، متغیر t زنده نخواهدبود.
			\item
سپس دستور \lr{z = a * 2;} حذف میگردد. زیرا بعد از آن، متغیر z زنده نخواهدبود.
		\end{enumerate}
نتیجه این تغییرات و آنالیز دوباره متغیرهای زنده به شکل زیر خواهد بود:
		\begin{latin}
			\begin{table}[H]
				\begin{tabular}{c|c}
					Instruction & Live Variables\\
					\hline
					a = 1 + 2; & {b, e} \\
					b = a + b; & {a, b, e}\\
					c = b + e; & {a, b, e}\\
					d = c + b; & {a, b, c}\\
					x = b + 3; & {a, b, d}\\
					z = a * 8; & {a, d, x}\\
					y = x - 2;  & {d, x, z}\\
					d = d - y;  & {d, x, y, z}\\
							& {d, x, z}\\
				\end{tabular} 
			\end{table} 
		\end{latin}
در دور دوم، هیچ دستوری با لحاظ کردن متغیرهای زنده حذف نمی‌شود. پس به سراغ مقادیر ثابت میرویم و \lr{Copy Propagation}
		\begin{enumerate}
			\item
ابتدا \lr{a = 1 + 2;} تبدیل میشود به \lr{a = 3;}
			\item
سپس \lr{b = a + b;} تبدیل میشود به \lr{b = 3 + b;}
			\item
سپس \lr{z = a * 8;} تبدیل میشود به \lr{z = 24;}
		\end{enumerate}
نتیجه این تغییرات و آنالیز دوباره متغیرهای زنده به شکل زیر خواهد بود:
		\begin{latin}
			\begin{table}[H]
				\begin{tabular}{c|c}
					Instruction & Live Variables\\
					\hline
					a = 3; & {b, e} \\
					b = 3 + b; & {b, e}\\
					c = b + e; & {b, e}\\
					d = c + b; & {b, c}\\
					x = b + 3; & {b, d}\\
					z = 24; & {d, x}\\
					y = x - 2;  & {d, x, z}\\
					d = d - y;  & {d, x, y, z}\\
							& {d, x, z}\\
				\end{tabular} 
			\end{table} 
		\end{latin}
در دور سوم و آخر، با توجه به متغیرهای زنده، فقط میتوان اولین جمله را حذف کرد. یعنی \lr{a = 3;}
\newline
کد نهایی:
	\lr{\lstinputlisting[]{commons/Q6_2.java}}
\end{enumerate}
}

