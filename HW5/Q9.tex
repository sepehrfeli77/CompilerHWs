%نام و نام خانوادگی:
%شماره دانشجویی: 
\مسئله{ }

\پاسخ{
\begin{itemize}
\item
	\begin{table}[H]
	\begin{tabular}{|l|l|l|l|l|l|l|l|l|l|}
		\hline
		variable & A & B & C & D & E & F & G & H & I \\
		\hline
		refcount & 1 & 0 & 1 & 0 & 1 & 1 & 1 & 2 & 1 \\ 
		\hline
		\end{tabular}
	\end{table}
\item
	\begin{table}[H]
	\begin{tabular}{|l|l|l|l|l|l|l|l|l|l|}
		\hline
		variable & A & B & C & D & E & F & G & H & I \\
		\hline
		refcount & 1 & 0 & 1 & 0 & 1 & 1 & 1 & 2 & 1 \\ 
		\hline
		\end{tabular}
	\end{table}
\item
به روش
\lr{stop and copy}
 عمل می‌کنیم و چون هیچ از A به سایر نودها وجود نخواهد داشت، نودهای باقی‌مانده CEFGHI (که به هم ارجاع دارند و یک دورمانند ایجاد می‌کنند) در این روش حذف می‌شوند و خانه‌ی A به سمت \lr{old space} منتقل می‌شود پس مقدار آدرس A به اندازه طول هر سمت کم یا زیاد می‌شود(بسته با این که
 \lr{old space} 
 در سمت چپ یا راست باشد). 
\end{itemize}
}

